\documentclass{article}

\usepackage{pgfplots}
\usepackage[margin=0.75in, paperwidth=8.5in, paperheight=11in]{geometry}
\usepackage{setspace}
\usepackage{fancyvrb} % extended verbatim environments
\usepackage{framed}%To get shade behind text

\definecolor{shadecolor}{rgb}{0.9,0.9,0.9}%setting shade color


\begin{document}
\pagenumbering{gobble}

\doublespacing
\textbf{IB Computer Science }                        %%%(class number and section) 
 \hfill                             %%%(date of test)
$ {\bf Name: } Young Prakseth{\hspace{2.5in}}$(3 points)

\begin{centering}
\vspace{1cm}
\textbf{Spring: Exam 1}\\
\end{centering}
\vspace{1cm}
 

  
 
 $\bf{1)}$ Write a Java class that prints the following. (10 points)
  
   \begin{verbatim}
 Hola mundo! 
 
  public class Halo{
  
          public static void main(String[] args){
                 System.out.println("Halo mundo!");
          }
     }         
 \end{verbatim}
 
 $\bf{2)}$ Write a method named "counter" that takes two integers, A and B, and prints the numbers from A to B. (20 points)
  \vspace{0.5cm} 
  
   \begin{verbatim}

  public class counter{
          
           public void Counter(int a,int b){

   int result = 0;
    while (a <=b){
    
        result+=a;
        a++; 
    }   

    System.out.println("The sum of all numbers is "+result); 
 }
 
   \end{verbatim}
   
 
  $\bf{3)}$ Write some Java code that will fill an array with the numbers from 10 to 100.  (20 points)
   \vspace{0.5cm}
   
   \begin{verbatim}

   public class ArrayPractice
{
       public static void main(String[] args)
    {   
        Random random = new Random();
        int[] a = new int[10];
        int i;

        for (i = 0; i < 10; i++)
        { 

            a[i] = 1 + random.nextInt(100);

               System.out.print(a[i]+ " ");

    }

}

\end{verbatim}

  $\bf{4)}$ Write a method named "average" that will return the average value of an integer array. \\
   It should return a double.  (20 points)
   \vspace{0.5cm}
   
     \begin{verbatim}

   public static double calcAverage() {
        
        int sum =0;
        
        for (int i=0; i < people.lenght; i++)
        
                    sum = sum + people[i];
                    
        double calcAverage() = sum / people.lenght
        
              System.out.println(people.clacAverage());
     }     
         \end{verbatim}
         
  $\bf{5)}$ Draw the truth table for OR and XOR.  (10 points)
   \vspace{0.5cm}
   
  $\bf{6)}$ Write a method for the XOR operator named "xor".  It should take two booleans as arguments and return a boolean.  (20 points)
   \vspace{0.5cm}
   
   $\bf{7)}$ Explain how to compile and run a program "hello.java" from the command line.  (10 points)
  


  
  
  
    

 
\end{document}